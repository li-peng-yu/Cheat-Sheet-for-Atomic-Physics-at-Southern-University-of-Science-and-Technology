\documentclass[10pt, a4paper, twocolumn]{article}

% 宏包引入
\usepackage[utf8]{inputenc}
\usepackage[T1]{fontenc}
\usepackage{ctex}           
\usepackage{amsmath}        
\usepackage{amssymb}        
\usepackage{mathtools}      
\usepackage{geometry}       
\usepackage{xcolor}         
\usepackage{booktabs}       
\usepackage{array}          
\usepackage{enumitem}       
\usepackage{titlesec} 
\usepackage{graphicx}

% --- 核心修改 1: 极致的页面边距 ---
\geometry{left=0.8cm, right=0.8cm, top=1cm, bottom=1cm}

% --- 核心修改 2: 全局紧凑设置 ---
\setlength{\parindent}{0pt}          
\setlength{\parskip}{0pt}            % 段落间不留空
\renewcommand{\baselinestretch}{0.95} %稍微压缩行距(不影响阅读)

% --- 核心修改 3: 压缩公式垂直间距 (这是省空间的大头) ---
\makeatletter
\g@addto@macro\normalsize{%
  \setlength\abovedisplayskip{2pt}
  \setlength\belowdisplayskip{2pt}
  \setlength\abovedisplayshortskip{0pt}
  \setlength\belowdisplayshortskip{0pt}
}
\makeatother

% --- 核心修改 4: 压缩章节标题间距 ---
\titleformat{\section}{\bfseries\small}{\thesection}{0em}{}[\hrule] % 标题更紧凑,加下划线区分
\titlespacing*{\section}{0pt}{6pt}{2pt} % 左边距,上方间距,下方间距

% --- 核心修改 5: 压缩列表间距 ---
\setlist[itemize]{nosep, leftmargin=*} % 列表无间距

% 自定义命令
\newcommand{\unit}[1]{\,\mathrm{#1}} 

\begin{document}

% --- 核心修改 6: 手动精简标题,替代 \maketitle ---
\twocolumn[
  \begin{center}
    \textbf{\large Atomic Physics Cheating Paper} \\
    \vspace{0.3em}
  \end{center}
]

%----------------------------------------------------------------
\section*{1. 卢瑟福散射与经典力学}

\begin{equation*}
    a \equiv \frac{1}{4\pi\varepsilon_0} \frac{Z_1 Z_2 e^2}{E}, \quad b = \frac{a}{2} \cot \frac{\theta}{2}
\end{equation*}

\textbf{粒子计数公式:}
\begin{equation*}
    dN' = N n t \left( \frac{Z_1 Z_2 e^2}{16\pi\varepsilon_0 E_k} \right)^2 \frac{d\Omega}{\sin^4(\frac{\theta}{2})}
\end{equation*}

\textbf{微分散射截面 (Rutherford):}
\begin{equation*}
    \frac{d\sigma}{d\Omega} = \sigma_{\mathrm{C}}(\theta) = \left( \frac{1}{4\pi\varepsilon_0} \frac{Z_1 Z_2 e^2}{4E} \right)^2 \frac{1}{\sin^4 \frac{\theta}{2}}
\end{equation*}

\textbf{实验室系截面:}
\begin{align*}
    \sigma_{\mathrm{L}}(\theta_{\mathrm{L}}) &= \left( \frac{1}{4\pi\varepsilon_0} \frac{Z_1 Z_2 e^2}{2 E_{\mathrm{L}} \sin^2 \theta_{\mathrm{L}}} \right)^2 \\
    &\quad \times \frac{\left[ \cos \theta_{\mathrm{L}} + \sqrt{1 - \left( \frac{m_1}{m_2} \sin \theta_{\mathrm{L}} \right)^2} \right]^2}{\sqrt{1 - \left( \frac{m_1}{m_2} \sin \theta_{\mathrm{L}} \right)^2}}
\end{align*}

\textbf{散射分数 $f$:}
\begin{equation*}
    f = n_s \cdot 4\pi C \cot^2\left(\frac{\theta_0}{2}\right), \quad C = \left(\frac{Z_1 Z_2 e^2}{16\pi\varepsilon_0 E}\right)^2
\end{equation*}

%----------------------------------------------------------------
\section*{2. 热辐射与玻尔模型}

\textbf{普朗克公式:}
\begin{equation*}
    E(\nu,T)d\nu = \frac{8\pi h \nu^3}{c^3} \frac{1}{e^{h\nu/kT} - 1} d\nu
\end{equation*}

\textbf{玻尔半径:}
\begin{equation*}
    r_n = \frac{4\pi\varepsilon_0 \hbar^2}{Zm_e e^2} \cdot n^2
\end{equation*}

\textbf{能级与速度:}
\begin{equation*}
    E_n = -\frac{1}{2}Z^2 m_e c^2 \alpha^2 \frac{1}{n^2} \approx -\frac{Z^2 13.6}{n^2} \unit{eV}
\end{equation*}
\begin{equation*}
    v_n = \frac{Z}{n} \alpha c, \quad \alpha = 1/137
\end{equation*}

\textbf{原子核运动修正(约化质量):}
\begin{equation*}
    R_A = R_\infty \frac{1}{1 + m_e/m_A}
\end{equation*}

%----------------------------------------------------------------
\section*{3. 光电效应与基本常数}

\textbf{光电效应方程:}
\begin{equation*}
    \frac{1}{2}mv_{m}^{2} = h\nu - \phi
\end{equation*}

\textbf{常用常数与关系:}
\begin{itemize}
    \item $\hbar c \approx 197 \unit{eV}\cdot\unit{nm}$ (联系长度和能量)
    \item $e^2/4\pi\varepsilon_0 \approx 1.44 \unit{eV}\cdot\unit{nm}$ (电磁项)
    \item $m_e c^2 \approx 0.511 \unit{MeV}$ (电子静止能量)
\end{itemize}

\textbf{相对论效应:}
\begin{gather*}
    E = mc^2 = \frac{m_0 c^2}{\sqrt{1 - (v/c)^2}} \\
    E^2 = (pc)^2 + (m_0 c^2)^2
\end{gather*}

%----------------------------------------------------------------
\section*{4. 量子力学基础}

\textbf{德布罗意波长:} $\lambda = h/p$

\textbf{不确定性原理:} $\Delta x \Delta p_x \ge \frac{\hbar}{2}, \Delta E \Delta t \ge \frac{\hbar}{2}$

\textbf{概率密度:}
$P(\boldsymbol{r}, t) = |\psi(\boldsymbol{r}, t)|^2 = \psi^* \psi$

\textbf{薛定谔方程 (定态):}
\begin{equation*}
    \nabla^2 \psi + \frac{2m}{\hbar^2}(E - V)\psi = 0
\end{equation*}

\textbf{一维无限深方势阱:}
\begin{equation*}
    E_n = \frac{n^2 h^2}{8ma^2}, \quad \psi_n(x) = \sqrt{\frac{2}{a}} \sin(\frac{n\pi x}{a}) \quad (n\ge 1)
\end{equation*}

\textbf{一维谐振子:}
$E_n = (n + \frac{1}{2})\hbar\omega \quad (n\ge 0)$

\textbf{量子隧穿 (透射系数):}
\begin{equation*}
    T \approx e^{-2\sqrt{2m(V_0-E)} \cdot d / \hbar}
\end{equation*}

\textbf{算符与期望值:}
$\bar{f} = \int \psi^* \hat{f} \psi d\tau$
\begin{equation*}
    \hat{p} = -i\hbar \nabla, \quad \hat{H} = -\frac{\hbar^2}{2m}\nabla^2 + V, \quad \hat{L}_z = -i\hbar \frac{\partial}{\partial \phi}
\end{equation*}

%----------------------------------------------------------------
\section*{5. 磁矩与角动量}

\textbf{布拉格衍射:} $2d \sin \theta = n\lambda$

\textbf{磁矩公式:}
\begin{align*}
    \vec{\mu}_l &= - \frac{e}{2m_e} \vec{L} = -\gamma \vec{L} \\
    \vec{\mu}_s &= -g_s \frac{e}{2m_e} \vec{S}
\end{align*}

\textbf{幅值与投影:}
\begin{itemize}
    \item $|\vec{\mu}_l| = \sqrt{l(l+1)}\mu_B$
    \item $|\vec{\mu}_J| = g_J \sqrt{j(j+1)} \mu_B$
    \item $\mu_z = -m_J g_J \mu_B$
    \item $S = \sqrt{s(s+1)}\hbar$
\end{itemize}

\textbf{玻尔磁子:}
$\mu_B = \frac{e\hbar}{2m_e} \approx 9.27 \times 10^{-24} \unit{J/T}$

\textbf{磁场中的力:} $F_z = \mu_z \frac{\partial B}{\partial z}$

\textbf{朗德 g 因子:}
$g_l=1, \quad g_s \approx 2$
\begin{equation*}
    g_J = 1 + \frac{j(j+1) + s(s+1) - l(l+1)}{2j(j+1)}
\end{equation*}

%----------------------------------------------------------------
\section*{6. 外磁场效应}

\textbf{能级分裂:}
\begin{itemize}
    \item 弱磁场:$\Delta E = \mu_B B g_j m_j$
    \item 强磁场:$\Delta E = \mu_B B (m_l + 2m_s)$
\end{itemize}

\textbf{谱线分裂:}
\begin{equation*}
    \Delta \tilde{\nu} = \frac{(M_{\text{上}} g_{\text{上}} - M_{\text{下}} g_{\text{下}}) \cdot \mu_B B}{hc}
\end{equation*}

\textbf{单纯能量差与磁场:}
$\Delta E = \Delta U = \frac{hc \Delta \lambda}{\lambda^2}$
\begin{equation*}
    B = \frac{hc \Delta \lambda}{2\lambda^2 \mu_B g_J} \quad (\text{塞曼分裂})
\end{equation*}

\textbf{施特恩-格拉赫实验通用公式:}
\begin{align*}
    \Delta Z &= \frac{g_J \mu_B \left( \frac{\partial B}{\partial z} \right) d D}{2 E_k} \cdot \Delta m_J \\
             &= \textcolor{cyan}{\frac{g_J \mu_B \left( \frac{\partial B}{\partial z} \right) d D}{m v^2} \cdot \Delta m_J} \\
             &= \textcolor{orange}{\frac{g_J \mu_B \left( \frac{\partial B}{\partial z} \right) d D}{3 k_B T} \cdot \Delta m_J}
\end{align*}

%----------------------------------------------------------------
\section*{7. 精细结构与多电子原子}

\textbf{原子光谱项符号:} 
\begin{equation*}
    {}^{2S+1}L_J \quad (\text{宇称 } \pi = (-1)^{\sum l_i})
\end{equation*}
\begin{itemize}
    \item $2S+1$: 自旋多重度
    \item $L$ (总轨道): $0(S), 1(P), 2(D), 3(F), 4(G), 5(H)$
    \item $J$ (总角动量): $|L-S| \le J \le L+S$
\end{itemize}

\textbf{洪特规则 (确定基态顺序):}
\begin{enumerate}
    \item \textbf{最大 $S$}:自旋多重度 $2S+1$ 最大者能量最低。
    \item \textbf{最大 $L$}:在 $S$ 相同时,轨道角动量 $L$ 最大者能量最低。
    \item \textbf{确定 $J$} (自旋-轨道耦合):
    \begin{itemize}
        \item 电子数 $\le$ 半满:$J = |L-S|$ (最小 J 能量最低)
        \item 电子数 $>$ 半满:$J = L+S$ (最大 J 能量最低)
    \end{itemize}
\end{enumerate}

\textbf{氢原子精细结构:}
\begin{align*}
    E_{nj} = &-\frac{13.6 Z^2}{n^2} \left[ 1 + \frac{(Z\alpha)^2}{n^2} \left( \frac{n}{j+1/2} - \frac{3}{4} \right) \right]
\end{align*}

\textbf{L-S 耦合能量项:}
\begin{equation*}
    \Delta E_{ls} = \frac{1}{2} A [J(J+1) - L(L+1) - S(S+1)]
\end{equation*}
其中 $A \propto \frac{Z^4}{n^3 l(l+1/2)(l+1)}$ (Landé interval rule)

\textbf{空穴原理:} 超过半充满壳层的电子数等效于空穴数($L, S$ 相同,但 $J$ 的基态规则相反)。

\textbf{泡利不相容与填充限额:}
\begin{itemize}
    \item 主壳层 $2n^2$; 分壳层 $2(2l+1)$
    \item 等效电子($n, l$ 相同)需考虑泡利原理排除重复态。
\end{itemize}

\begin{table}[h]
    \centering
    \small
    \renewcommand{\arraystretch}{1.1}
    \setlength{\tabcolsep}{2pt}
    \begin{tabular}{|l|l|}
        \hline
        \textbf{类型} & \textbf{计算步骤} \\
        \hline
        \textbf{L-S} & 
        1. $\vec{L} = \sum \vec{l}_i$ \quad
        2. $\vec{S} = \sum \vec{s}_i$ \quad
        3. $\vec{J} = \vec{L} + \vec{S}$ \\
        \hline
        \textbf{j-j} & 
        1. $\vec{j}_i = \vec{l}_i + \vec{s}_i$ \quad
        2. $\vec{J} = \sum \vec{j}_i$ \\
        \hline
    \end{tabular}
\end{table}

%----------------------------------------------------------------
\section*{8. X 射线与康普顿散射}

\textbf{短波限:} $\lambda_{min} = hc/E = hc/eV$

\textbf{康普顿波长偏移:}
$\Delta \lambda = \lambda' - \lambda = \frac{h}{m_e c}(1 - \cos \theta)$

\textbf{衰减定律:} $I = I_0 e^{-\mu x}$

\textbf{能量转移 (电子获得动能):} $E_k' = E - E'$

\textbf{莫塞莱定律:}
\begin{equation*}
    E_0 = 13.6 \unit{eV} \cdot (Z - \sigma)^2 \cdot \left( \frac{1}{n_1^2} - \frac{1}{n_2^2} \right)
\end{equation*}

\textbf{散射光子能量 (康普顿):}
\begin{equation*}
    E' = \frac{E}{1 + \frac{E}{m_p c^2}(1 - \cos \theta)}
\end{equation*}

%----------------------------------------------------------------
\section*{9. 径向波函数附录}

\textbf{积分公式:} $\int_0^\infty x^n e^{-\alpha x} dx = \frac{n!}{\alpha^{n+1}}$

\textbf{类氢原子径向波函数 $R_{n,l}(r)$:}

{\footnotesize
\begin{align*}
    R_{1,0} &= 2 \left( \frac{Z}{a_1} \right)^{3/2} e^{-Zr/a_1} \\
    R_{2,0} &= 2 \left( \frac{Z}{2a_1} \right)^{3/2} \left( 1 - \frac{Zr}{2a_1} \right) e^{-Zr/2a_1} \\
    R_{2,1} &= \frac{1}{\sqrt{3}} \left( \frac{Z}{2a_1} \right)^{3/2} \left( \frac{Zr}{a_1} \right) e^{-Zr/2a_1} \\
    R_{3,0} &= 2 \left( \frac{Z}{3a_1} \right)^{3/2} \left[ 1 - \frac{2Zr}{3a_1} + \frac{2}{27} \left( \frac{Zr}{a_1} \right)^2 \right] e^{-\frac{Zr}{3a_1}} \\
    R_{3,1} &= \frac{4\sqrt{2}}{3} \left( \frac{Z}{3a_1} \right)^{3/2} \left[ \frac{Zr}{a_1} - \frac{1}{6} \left( \frac{Zr}{a_1} \right)^2 \right] e^{-\frac{Zr}{3a_1}} \\
    R_{3,2} &= \frac{2\sqrt{2}}{27\sqrt{5}} \left( \frac{Z}{3a_1} \right)^{3/2} \left( \frac{Zr}{a_1} \right)^2 e^{-Zr/3a_1}
\end{align*}
}

\vfill % 1. 填充垂直空间,将后续内容推到底部

\begin{center} % 2. 居中环境
    % 3. 插入图片:width=0.8\linewidth 表示占当前栏宽度的 80%
    % 请确保你已将图片上传并重命名为 god_bless_atomic.jpg (或实际文件名)
    \includegraphics[width=1.95\linewidth]{atomic.png} 
\end{center}

\end{document}
